\section{Tema}

\paragraph{}Pessoas acometidas por acidente vascular cerebral (AVC) desenvolvem sequelas cognitivas (impedimento de fala, problemas de concentração) e físicas (espasticidade, paralisia parcial). O trabalho focará em apenas uma sequela física, a espasticidade.

\paragraph{}Para o tratamento da espasticidade, o paciente deve passar por sessões de fisioterapia, aonde realizará movimentos com a ajuda do médico. Com o avanço da tecnologia, novas formas de tratamento surgiram, se baseando principalmente na eletroestimulação neuromuscular (NMES).

\paragraph{}Para a realização de tal tratamento de maneira eficiente, é necessário um dispositivo auxiliar que seja capaz de se adaptar aos diferentes tipos de pacientes com variados níveis e tipos de lesões.

\paragraph{}Este trabalho então, estuda e avalia a eficácia de diferentes algoritmos de controle para o tratamento de espasticidade provinda do AVC através de eletroestimulação neuromuscular.


\section{Delimitação}

\paragraph{}------------------------------------------------------


\section{Justificativa}

\paragraph{}Tais dispositivos auxiliares discutidos anteriormente são escassos e caros no Brasil (existem apenas alguns importados) e destes, nenhum possui a realimentação do paciente, ele é operado em malha aberta, o que requer um conhecimento específico do médico (diminuindo a capacidade de atuação do aparelho) e diminui a eficácia do tratamento. Portanto o desenvolvimento deste dispositivo poderia ajudar na recuperação de milhares de pacientes da maneira mais eficaz possível.

\section{Objetivos}

\paragraph{}O objetivo deste trabalho é estudar e desenvolver técnicas de controle robustas e suficiente adaptativas para utilização com NMES voltada para fisioterapia no tratamento de espasticidade. No futuro, com a realização e implementação bem sucedida deste, se espera um aparelho autônomo, que não precise de calibração por parte do médico, e se adapte à toda (ou quase toda) situação clínica no tratamento da espasticidade.


\section{Metodologia}

\paragraph{}Como é a abordagem do assunto. Como foi feita a pesquisa, se vai houve validação, etc. Em resumo, você de explicar qual foi sua estratégia para atender ao objetivo do trabalho (tamanho do texto: livre).


\section{Descrição}

\paragraph{}No capítulo 2 será feito uma revisão de todo o conteúdo necessário para o entendimento do trabalho, especialmente dos termos médicos e técnicos que serão usados posteriormente.

\paragraph{}O capítulo 3 apresenta ...

\paragraph{}Os .... são apresentados no capítulo 4. Nele será explicitado ...

\paragraph{}E assim vai até chegar na conclusão.
